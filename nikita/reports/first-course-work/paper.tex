\documentclass[aps,%
12pt,%
final,%
oneside,
onecolumn,%
musixtex, %
superscriptaddress,%
centertags]{article} %%
\topmargin=-40pt
\textheight=650pt
\usepackage[english,russian]{babel}
\usepackage[utf8]{inputenc}
%всякие настройки по желанию%
\usepackage[colorlinks=true,linkcolor=blue,unicode=true]{hyperref}
\usepackage{euscript}
\usepackage{supertabular}
\usepackage[pdftex]{graphicx}
\usepackage{amsthm,amssymb, amsmath}
\usepackage{textcomp}
\usepackage[noend]{algorithmic}
\usepackage[ruled]{algorithm}
\selectlanguage{russian}

\begin{document}

\begin{titlepage}
\begin{center}
% Upper part of the page
\textbf{\Large САНКТ-ПЕТЕРБУРГСКИЙ \\ ГОСУДАРСТВЕННЫЙ УНИВЕРСИТЕТ} \\[1.0cm]
\textbf{\large Математико-Механический факультет} \\[0.2cm]
\textbf{\large Кафедра информационно-аналитических систем}\\[3.5cm]

% Title
\textbf{\LARGE Применение машинного обучения для анализа угольных скважин}\\[1.0cm]
\textbf{\Large Дипломная работа студента 546 группы} \\[0.2cm]
\textbf{\Large Чурикова Никиты Сергеевича} \\[3.5cm]

%supervisor
\begin{flushright} \large
\emph{Научный руководитель:} \\
Доцент \textsc{Графеева Н. Г.}
\end{flushright}
\begin{flushright} \large
\emph{Заведующий кафедрой:} \\
Доцент \textsc{Михайлова Е. Г.}
\end{flushright}
\vfill

% Bottom of the page
{\large {Санкт-Петербург}} \par
{\large {2017 г.}}
\end{center}
\end{titlepage}

% Table of contents
\tableofcontents

\section{Введение}
Машинное обучение проникает во многие сферы нашей жизни~\cite{overview-of-ml} автоматизируя
различные рутинные процессы, вроде поездок на машине~\cite{ai-cars} и обработки рутинных документов.
Поэтому у профессионалов из различных областей естественно возникает желание сократить время работы
на не столь увлекательных задачах.

В данном тексте пойдет речь о применении машинного обучения в области геофизики. У специалистов
в этой области есть очень трудоемкая задача по выделению литологии на различной глубине в почве.
Будет показано, что представляют из себя данные скважин, которые геофизики анализируют, какие
наработки, продукты и технологии в данной области уже есть, а также будут приведены наработки и
идеи автора по данной задаче.

\section{Обзор литературы}
Начать разбираться в области применения машинного обучения к классификации литологии
стоит с соревнования по данному вопросу~\cite{SEG-contest}, которое проводилось
сообществом SEG~\cite{SEG}. В этом контесте приводят отличный пример по тому, как начинать
с работать данными скважин, они выкладывают открытый датасет, на котором можно потренироваться,
а также объясняют и показывают что литологии можно спутать с их "соседями", т.е. литологиями,
которые трудно различить между собой даже геофизикам.

Также по результатам этого соревнования были написаны интересные статьи.
Эти замечательные работы кратко описывают научные результаты контеста. Статья~\cite{Bestagini2017a}
подводит итоги и рассказывает о том, как генерировать новые атрибуты используя глубину, а также
показывает, что лучшим алгоритмом соревнования были деревья основанные на градиентном бустинге~\cite{xgboost}.
В статье~\cite{Tschannen2017} приведена попытка применить популярный алгоритм convolutional
neural network (CNN)~\cite{cnn}. Несмотря на то что они популярны и то что атрибуты
являются вещественными значениями, на этих данных алгоритм не попал даже в десятку лучших решений.
Авторы статьи утверждают, что проблема заключается в недостаточном количестве данных.

Неплохой литературой для начала погружения в геофизику и машинное обучение является книга
Мухамедиева~Р.И.~\cite{GeophysicsMLBook}. В этой работе приведено хороше описание методов
каротажа, базовых алгоритмов машинного обучения, а также приводятся рекомендации по подготовке
таких специфичных данных. В частности они не рекомендуют использовать вейвлет
преобразования~\cite{wavelet}, а советуют обратить внимание на следующие этапы предобработки данных:
\begin{enumerate}
  \item Удаление аномальных значений;
  \item Линейная нормировка;
  \item Очистка данных по методу «ближайших соседей»;
  \item Формирование плавающего окна данных.
\end{enumerate}





% \subsection{Мотивация}
% \subsection{Постановка задачи}
% \subsection{Доступные программные средства}
% \subsection{Полученные результаты}
%
% \section{Основная часть раз}
% Секций в основной части может быть сколько угодно.
%
% \section{Основная часть два: Теория}
%
% \section{Основная часть два: Детали реализации}
% \subsection{Расчётная часть}
%
% \section{Анализ экспериментов.}
% \begin{figure}[ht]
% \begin{center}
%
% \scalebox{0.4}{
%    \includegraphics{images/graph.jpg}
% }
%
% \caption{
% \label{graph-fig}
%      Линейные функции.}
% \end {center}
% \end {figure}
% Ссылаемся на график ~\ref{graph-fig}.
% Ссылка на статью: \cite{DBLP:conf/adbis/NovikovP03}
\section{Заключение}

%ручной ввод библиографии.
%для тестирования, убрать комментарий
%\begin{thebibliography}{}

%\bibitem{voc} Griffin D.W., Lim J.S. \flqq Multiband excitation vocoder\frqq. IEEE ASSP-36 (8), 1988, pp. 1223-1235.
%\end{thebibliography}

%автоматическая генерация библиографии из бибтеховского файла. Из файла подгрузятся только те статьи, на которые есть ссылки в тексте!
\bibliographystyle{gost780s}
\bibliography{test}

\end{document}
